\documentclass[11pt]{article}

\usepackage{graphicx}
\usepackage{amsmath,amsfonts,amssymb}

\usepackage{hyperref}  % for urls and hyperlinks


\setlength{\textwidth}{6.2in}
\setlength{\oddsidemargin}{0.3in}
\setlength{\evensidemargin}{0in}
\setlength{\textheight}{9in}
\setlength{\voffset}{-1in}
\setlength{\headsep}{26pt}
\setlength{\parindent}{0pt}
\setlength{\parskip}{5pt}

% input some useful macros from RJLmacros.tex:
\input{../hw1/RJLmacros}

\begin{document}

\hfill\vbox{\hbox{AMath 574}\hbox{Homework 2}
\hbox{Due by 11:00pm on January 26, 2023}}

For submission instructions, see:

\url{http://faculty.washington.edu/rjl/classes/am574w2023/homework2.html}

%--------------------------------------------------------------------------
\vskip 1cm
\hrule
{\bf Problem \#3.4 in the book}

Also sketch the solution in the $x$--$t$ plane, labeling the states that
appear in the solution. Then show where these states are in the $p$--$u$
phase plane and how they are connected by eigenvectors.

Note that the script \verb+problem_3_5.py+ was used to generate the figure
for the next problem and could be simplified for this problem if desired.

% uncomment the next two lines if you want to insert solution...
%\vskip 1cm
%{\bf Solution:}

% insert your solution here!


%--------------------------------------------------------------------------

\vskip 1cm
\hrule
{\bf Problem \#3.5 in the book}

The script \verb+problem_3_5.py+ was used to generate this figure:

% you need to run
%     python problem_3_5.py
% at the command line to generate the figure inserted here:
\hfil\includegraphics[width=3.5in]{problem_3_5.png}\hfil

To solve this problem, determine the states $A,~ B, ~ \ldots,~ M$ and also
the times $t_1,~t_2,~t_3$.  The times can be written in terms of the
parameters $\rho_0$ and $K_0$, which were not stated in the problem.

For example,
\[
A = \bcm 0 \\ 0 \ecm, \quad B = \bcm 1 \\ 0 \ecm, \quad
C = \bcm 0 \\ 0 \ecm, \quad \ldots
\]

% Note that bcm and ecm (begin and end centered matrices) are defined in 
% RJLmacros.tex

{\em Hint:} Another way to think about what happens when waves reflect off
the boundary is suggested in Section 7.3.3.


% uncomment the next two lines if you want to insert solution...
%\vskip 1cm
%{\bf Solution:}

% insert your solution here!



%--------------------------------------------------------------------------

\vskip 1cm
\hrule
{\bf Problem \#3.8 in the book}


% uncomment the next two lines if you want to insert solution...
%\vskip 1cm
%{\bf Solution:}

% insert your solution here!

%--------------------------------------------------------------------------

\vskip 1cm
\hrule
{\bf Problem \#4.2 in the book}


% uncomment the next two lines if you want to insert solution...
%\vskip 1cm
%{\bf Solution:}

% insert your solution here!


%--------------------------------------------------------------------------

\vskip 1cm
\hrule
{\bf Problem \#4.3 in the book}


% uncomment the next two lines if you want to insert solution...
%\vskip 1cm
%{\bf Solution:}

% insert your solution here!



%--------------------------------------------------------------------------
\vskip 1cm
\hrule
{\bf Programming Problem.}

This will involve modifying a Jupyter notebook for solving the acoustics equations using Godunov's method and another method.  I will get this posted by Friday 1/20 and will also discuss in class.


%--------------------------------------------------------------------------

\end{document}
