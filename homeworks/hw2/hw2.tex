\documentclass[11pt]{article}

\usepackage{graphicx}
\usepackage{amsmath,amsfonts,amssymb}

\usepackage{hyperref}  % for urls and hyperlinks


\setlength{\textwidth}{6.2in}
\setlength{\oddsidemargin}{0.3in}
\setlength{\evensidemargin}{0in}
\setlength{\textheight}{9in}
\setlength{\voffset}{-1in}
\setlength{\headsep}{26pt}
\setlength{\parindent}{0pt}
\setlength{\parskip}{5pt}

% input some useful macros from RJLmacros.tex:
\input{../hw1/RJLmacros}

\begin{document}

\hfill\vbox{\hbox{AMath 574}\hbox{Homework 2}
\hbox{Updated 1/21/23 to include final problem}
\hbox{Due by 11:00pm on January 26, 2023}}

For submission instructions, see:

\url{http://faculty.washington.edu/rjl/classes/am574w2023/homework2.html}

%--------------------------------------------------------------------------
\vskip 1cm
\hrule
{\bf Problem \#3.4 in the book}

Also sketch the solution in the $x$--$t$ plane, labeling the states that
appear in the solution. Then show where these states are in the $p$--$u$
phase plane and how they are connected by eigenvectors.

Note that the script \verb+problem_3_5.py+ was used to generate the figure
for the next problem and could be simplified for this problem if desired.

% uncomment the next two lines if you want to insert solution...
%\vskip 1cm
%{\bf Solution:}

% insert your solution here!


%--------------------------------------------------------------------------

\vskip 1cm
\hrule
{\bf Problem \#3.5 in the book}

The script \verb+problem_3_5.py+ was used to generate this figure:

% you need to run
%     python problem_3_5.py
% at the command line to generate the figure inserted here:
\hfil\includegraphics[width=3.5in]{problem_3_5.png}\hfil

To solve this problem, determine the states $A,~ B, ~ \ldots,~ M$ and also
the times $t_1,~t_2,~t_3$.  The times can be written in terms of the
parameters $\rho_0$ and $K_0$, which were not stated in the problem.

For example,
\[
A = \bcm 0 \\ 0 \ecm, \quad B = \bcm 1 \\ 0 \ecm, \quad
C = \bcm 0 \\ 0 \ecm, \quad \ldots
\]

% Note that bcm and ecm (begin and end centered matrices) are defined in 
% RJLmacros.tex

{\em Hint:} Another way to think about what happens when waves reflect off
the boundary is suggested in Section 7.3.3.


% uncomment the next two lines if you want to insert solution...
%\vskip 1cm
%{\bf Solution:}

% insert your solution here!



%--------------------------------------------------------------------------

\vskip 1cm
\hrule
{\bf Problem \#3.8 in the book}


% uncomment the next two lines if you want to insert solution...
%\vskip 1cm
%{\bf Solution:}

% insert your solution here!

%--------------------------------------------------------------------------

\vskip 1cm
\hrule
{\bf Problem \#4.2 in the book}


% uncomment the next two lines if you want to insert solution...
%\vskip 1cm
%{\bf Solution:}

% insert your solution here!


%--------------------------------------------------------------------------

\vskip 1cm
\hrule
{\bf Problem \#4.3 in the book}


% uncomment the next two lines if you want to insert solution...
%\vskip 1cm
%{\bf Solution:}

% insert your solution here!



%--------------------------------------------------------------------------
\vskip 1cm
\hrule
{\bf Programming Problem.}

The notebook in the class repository at 
{\tt \$AM574/notebooks/acoustics\_godunov.ipynb}
has been updated since class on 1/20 to make it a bit easier to modify for
this problem.  Please make sure you ``git pull'' to get the latest version.
And please contact me if you are having problems with Jupyter of Python!
We will do more with this same notebook in future homeworks.

\vskip 10pt
(a) Implement a new function {\tt LxF\_step} similar to {\tt Godunov\_step}
that takes a single step with the Lax-Friedrichs methods (for the same
acoustics system):  
\[
\qinp =  \qin - \frac{\Dt}{2\Dx} A(\qipn-\qimn)
\]

Since {\tt Pn} and {\tt Un} are separate arrays, you
will have to implement the matrix-vector products implied in Lax-Friedrichs
componentwise.  

You might want to test with Courant number 1 first when debugging.

Submit your notebook that contains the implementation and also does at least
the experiment described in part (b) below. You can include other tests too
if you want.


\vskip 10pt
(b) When you do the following experiment:

\hfil\includegraphics[width=6.0in]{input.png}\hfil

you should get plots like theese:

\hfil\includegraphics[width=3.5in]{plot1.png}\hfil

\vskip 10pt
\hfil\includegraphics[width=3.5in]{plot2.png}\hfil

Explain why there are always two adjacent points with exactly the same
value, for this particular choice of initial data.

\vskip 10pt
Note that that I inserted {\tt savefig} commands in the input to create the
png files used here.

See \url{https://matplotlib.org/stable/api/_as_gen/matplotlib.pyplot.savefig.html}
for more about this command.


%--------------------------------------------------------------------------

\end{document}
