\documentclass[11pt]{article}

\usepackage{graphicx}
\usepackage{amsmath,amsfonts,amssymb}

\usepackage{hyperref}  % for urls and hyperlinks


\setlength{\textwidth}{6.2in}
\setlength{\oddsidemargin}{0.3in}
\setlength{\evensidemargin}{0in}
\setlength{\textheight}{9in}
\setlength{\voffset}{-1in}
\setlength{\headsep}{26pt}
\setlength{\parindent}{0pt}
\setlength{\parskip}{5pt}

% input some useful macros from RJLmacros.tex:
\input{RJLmacros}

\begin{document}

\hfill\vbox{\hbox{AMath 574}\hbox{Homework 1}
\hbox{Due by 11:00pm on January 17, 2023 (updated)}}

For submission instructions and some additional information, see:

\url{http://faculty.washington.edu/rjl/classes/am574w2023/homework1.html}


%--------------------------------------------------------------------------
\vskip 1cm
\hrule
{\bf \#1.}

The gas dynamics equations (2.38) are written as conservation laws for the
mass and momentum, which are conserved quantities. Often the so-called
``primitive variables'' $p(x,t)=$ pressure and $u(x,t)=$ velocity are more
natural to use, e.g. in the acoustic equations we consider perturbations of
$p$ and $u$ modeled by the linear system (2.5), rather than the
linear system with matrix (2.46) modeling perturbations in mass and momentum.

(a) Starting from (2.38), derive the following nonlinear equations for the
pressure and velocity:
\begin{equation}\label{primeqn}
\begin{split}
p_t + up_x + \rho P'(\rho) u_x &= 0,\\
u_t + (1/\rho)p_x + uu_x &= 0.
\end{split} 
\end{equation} 
Note that these equations involve $\rho(x,t)$ which must now be determined
from $p(x,t)$ using the equation of state, by inverting $p = P(\rho)$ for
$\rho$ as a function of $p$.  (Exactly what this gives depends on the
particular equation of state, so you don't need to go farther.)

{\em Hint:} Use for example $(\rho u)_t = \rho_t u + \rho u_t$ in the
conservation law for momentum and then use the conservation law for
$\rho_t$, and also note $P(\rho)_x = P'(\rho)\rho_x,$ etc.

(b) The equations (\ref{primeqn}) can be written in the form 
\[
q_t(x,t) + A(q(x,t)) q_x(x,t) = 0,
\]
a non-conservative nonlinear system.  Determine the matrix $A(q)$ and
the eigenvalues of the matrix.  

The system is said to be hyperbolic if the matrix is diagonalizable
with real eigenvalues. Confirm that the resulting condition on
$P'(\rho)$ (and the eigenvalues) agree with what we found from the
conservative form (2.38).  
Also note that if we linearize (\ref{primeqn}) about
\[
p\approx p_0,~~ u\approx u_0,~~ \rho P'(\rho) \approx \rho_0
P'(\rho_0)\equiv K_0,
\]
then the linearized equations agree with the acoustics equations (2.50).

{\bf Note:} This gives a nice derivation of the linear acoustics equations,
but when solving a nonlinear gas dynamics problem it is generally necessary
to use the conservative form in order to get proper modeling of shock waves
and other nonlinear phenomena.  Smooth solutions should agree between the
two formulations.

% uncomment the next two lines if you want to insert solution...
%\vskip 1cm
%{\bf Solution:}

% insert your solution here!


%--------------------------------------------------------------------------
\newpage
\vskip 1cm
\hrule
{\bf Problem \#2.7 in the book}


% uncomment the next two lines if you want to insert solution...
%\vskip 1cm
%{\bf Solution:}

% insert your solution here!


%--------------------------------------------------------------------------
\vskip 1cm
\hrule
{\bf Problem \#2.8 in the book}

{\em Hint:} Read Section 2.13 first and note that when linearizing about a
constant specific volume $V_0$, the relation (2.102) between $\xi$ and $x$ is
roughly $\xi = (x-x_0)/V_0$.


% uncomment the next two lines if you want to insert solution...
%\vskip 1cm
%{\bf Solution:}

% insert your solution here!


%--------------------------------------------------------------------------
\vskip 1cm
\hrule
{\bf Problem \#3.1(d,e,f) in the book} You might want to do Problem 3.2 first.


% uncomment the next two lines if you want to insert solution...
%\vskip 1cm
%{\bf Solution:}

% insert your solution here!


%--------------------------------------------------------------------------
\vskip 1cm
\hrule
{\bf Problem \#3.2 in the book}

You can use Matlab for this one, but I suggest you try writing the program 
in Python.  A Jupyter notebook with a partial solution can be found in the
class repository to help get you started.

Note that the module {\tt numpy.linalg} contains an {\tt eig}
function similar to Matlab.

% uncomment the next two lines if you want to insert solution...
%\vskip 1cm
%{\bf Solution:}

% insert your solution here!


%--------------------------------------------------------------------------

\vskip 1cm
\hrule
{\bf Problem \#3.3 in the book}

You do not need to draw the dashed lines of Figure 3.3, just the wedges with
the correct wave speeds.  Sketch it by hand or e.g. in a Jupyter notebook,
as you please.

% uncomment the next two lines if you want to insert solution...
%\vskip 1cm
%{\bf Solution:}

% insert your solution here!

% to insert a figure named X.png, you might use this...
% \hfil\includegraphics[width=4.0in]{X.png}\hfil

%--------------------------------------------------------------------------
\end{document}

